%%%%%%%%%%%%%%%%%%%%%%%%%%%%%%%%%%%%%%%%%
% Beamer Presentation
% LaTeX Template
% Version 1.0 (10/11/12)
%
% This template has been downloaded from:
% http://www.LaTeXTemplates.com
%
% License:
% CC BY-NC-SA 3.0 (http://creativecommons.org/licenses/by-nc-sa/3.0/)
%
%%%%%%%%%%%%%%%%%%%%%%%%%%%%%%%%%%%%%%%%%

%----------------------------------------------------------------------------------------
%	PACKAGES AND THEMES
%----------------------------------------------------------------------------------------

\documentclass{beamer}

\mode<presentation> {

% The Beamer class comes with a number of default slide themes
% which change the colors and layouts of slides. Below this is a list
% of all the themes, uncomment each in turn to see what they look like.

\usetheme{Madrid}

\setbeamertemplate{footline} % To remove the footer line in all slides uncomment this line
%\setbeamertemplate{footline}[page number] % To replace the footer line in all slides with a simple slide count uncomment this line

\setbeamertemplate{navigation symbols}{} % To remove the navigation symbols from the bottom of all slides uncomment this line
}

\usepackage{graphicx} % Allows including images
\usepackage{booktabs} % Allows the use of \toprule, \midrule and \bottomrule in tables

%----------------------------------------------------------------------------------------
%	TITLE PAGE
%----------------------------------------------------------------------------------------

\title[Conditional Conjectures in QuickSpec]{Conditional Conjectures in QuickSpec} % The short title appears at the bottom of every slide, the full title is only on the title page

\author{Maximilian Algehed} % Your name
\institute[CTH] % Your institution as it will appear on the bottom of every slide, may be shorthand to save space
{
Chalmers University of Technology \\ % Your institution for the title page
\medskip
\textit{m.algehed@gmail.com} % Your email address
}
\date{\today} % Date, can be changed to a custom date

\begin{document}

\begin{frame}
    \titlepage % Print the title page as the first slide
\end{frame}

\begin{frame}
    \frametitle{What is QuickSpec?}
        \Large{\centerline{Functions in, equations out!}}
\end{frame}

\begin{frame}
    \frametitle{Demo!}
        \Huge{\centerline{Demo!}}
\end{frame}

\begin{frame}
    \frametitle{How does it do this?}
        \begin{itemize}
            \item Generate schemas\\\texttt{? ++ (? ++ ?)}
            \item Instantiate schemas\\\texttt{xs ++ (ys ++ zs)}
            \item Test schemas against each other\\\texttt{xs ++ (ys ++ zs) = (xs ++ ys) ++ zs}
            \item QuickSpec tests the most general schemas first\\
                \texttt{xs ++ (ys ++ zs) = (xs ++ ys) ++ zs}\\implies\\\texttt{xs ++ (xs ++ xs) = (xs ++ xs) ++ xs}
            \item Uses Knuth-Bendix completion to determine if it needs to test a term at all
        \end{itemize}
\end{frame}

\begin{frame}
    \frametitle{What's this?}
        \Large{\centerline{\texttt{zip is (is ++ js) = zip is is}}}
\end{frame}

\begin{frame}
    \frametitle{What we'd like}
        \centerline{\texttt{length xs = length ys => zip xs (ys ++ js) = zip xs ys}}
\end{frame}

\begin{frame}
    \frametitle{Demo!}
        \Huge{\centerline{Demo!}}
\end{frame}

\begin{frame}
    \frametitle{How does it work?}
        What old QuickSpec does
        \begin{itemize}
            \item Run QuickSpec with the normal signature 
            \item Run QuickSpec once more for every condition 
        \end{itemize}
\end{frame}

\begin{frame}
    \frametitle{How does it work?}
        What old QuickSpec is
        \begin{itemize}
            \item Slow
            \item Noisy\\
                \texttt{P x => f y         = g x}\\
                \texttt{P x and Q y => f y = g x}
        \end{itemize}
        Only used in Hipster (and HipSpec).
\end{frame}

\begin{frame}[fragile]
    \frametitle{How does it work?}
        \begin{verbatim}
        data Peqlen = Peqlen {xs :: [Int],
                              ys :: [Int]}

        instance Arbitrary Peqlen where
            arbitrary = do
                            l  <- arbitrary 
                            xs <- sequence $
                                replicate l arbitrary
                            ys <- sequence $
                                replicate l arbitrary
                            return (Peqlen xs ys)
        \end{verbatim}
\end{frame}

\begin{frame}
    \frametitle{How does it work?}
        \centerline{\texttt{length xs = length ys => zip xs (ys ++ is) = zip xs ys}}
        \centerline{}
        \centerline{becomes}
        \centerline{}
        \centerline{\texttt{zip (xs p) (ys p ++ is) = zip (xs p) (ys p)}}
        \pause
        \centerline{}
        \centerline{\texttt{zip (ys p) (xs p ++ is) = zip (ys p) (xs p)}}
        \centerline{}
        \centerline{(\texttt{length} is a cornercase)}
\end{frame}

\begin{frame}
    \frametitle{So what about redundancy?}
        \Large{\centerline{\texttt{f z = g (x p)}}}
        \centerline{}
        \centerline{more general than}
        \centerline{}
        \Large{\centerline{\texttt{f (y q) = g (x p)}}}
\end{frame}

\begin{frame}[fragile]
    \frametitle{Generalising}
        \begin{verbatim}
        arbitrary = arbitrary `suchThat`
            (\(xs, ys) -> length xs == length ys) >>=
            return . (uncurry Peqlen) 
        \end{verbatim}   
\end{frame}

\begin{frame}
    \frametitle{Demo!}
        \Large{\centerline{Some more examples}}
\end{frame}

\begin{frame}
    \frametitle{Something more interesting}
        In APL, the $\rho$ operator gets the shape of a value.\\
        \begin{align*}
            \rho\;\begin{pmatrix} 0 & 2 \end{pmatrix}\; &= \begin{pmatrix}4\end{pmatrix}\\\\
            \rho \begin{pmatrix} 2 & 1 \\ 3 & 4 \end{pmatrix} &= \begin{pmatrix} 2 & 2\end{pmatrix}\\\\
            \rho \begin{pmatrix} 2 & 1 \\ 1 \end{pmatrix} &= crash...
        \end{align*}
\end{frame}

\begin{frame}
    \frametitle{Something more interesting}
        The $\times$ operator does element wise multiplication.\\
        \begin{align*}
            \begin{pmatrix} 1 & 2\\3 & 4\end{pmatrix} \times \begin{pmatrix} 2 & 3\\6 & 2\end{pmatrix} &=
            \begin{pmatrix} 2 & 6\\18 & 2\end{pmatrix}\\\\
            \begin{pmatrix} 1 & 4 & 3 \end{pmatrix} \times \begin{pmatrix}1 & 0 & 0 & 1 \end{pmatrix} &= crash...
        \end{align*}
\end{frame}

\begin{frame}
    \frametitle{Something more interesting}
    \centerline{What we expect}
    \begin{equation*}
        \rho (x \times y) = \rho x = \rho y
    \end{equation*}
    \centerline{}
    \centerline{for all well behaved x and y...}
\end{frame}

\begin{frame}
    \frametitle{Being well behaved}
    \begin{columns}
        \column{.5\textwidth}
            \begin{itemize}
                \item $\rho$ of $x$ does not crash
                \item $\rho$ of $y$ does not crash
                \item $\rho x = \rho y$
            \end{itemize}
        \column{.5\textwidth}
            \pause
            \centerline{\texttt{suchThat} won't cut it}
    \end{columns}
\end{frame}

\begin{frame}
    \frametitle{Demo!}
    \Huge{\centerline{Demo!}}
\end{frame}

\begin{frame}
    \Huge{\centerline{Questions?}}
\end{frame}

\end{document} 
