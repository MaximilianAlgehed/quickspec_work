\section{Different encodings}

    This section describes three
    ways of encoding conditional
    equations in QuickSpec. It is
    worth noting that these methods
    are general and could be used outside
    the context of QuickSpec. In fact,
    the idea of encoding conditionals
    as types is something users of
    QuickCheck, Hughes and Cleassen, %QuickCheck reference
    have used for a long time.

    \subsection{Type encoding}

        Type encoding is based on the idea of encoding
        predicates as dependent types. More specifically
        it is saying that for a given predicate
        \begin{verbatim}
            p :: (a, b, c, ..., x) -> Bool
        \end{verbatim}
        there exists a dependent type
        \begin{verbatim}
           Predicate p
        \end{verbatim}
        and a function 
        \begin{verbatim}
            data :: Predicate p -> (a, b, c, ..., d)
        \end{verbatim}
        such that 
        \begin{verbatim}
            p . data = const True
        \end{verbatim}
        This may seem odd when you consider that haskell is not a dependently types
        language and QuickSpec works in the domain of haskell.
        However, becuase QuickSpec uses QuickCheck and QuickCheck lets us define
        custom generators for random data we can write a generic generator for
        these types. This means that when equations involving these predicate
        types are tested by QuickCheck the types behave as if they were dependent types.
        The equations involving predicate types are on the form:
        \begin{verbatim}
            
            f (data p) = g (data q)

        \end{verbatim}
        which translates nicely to $p(x) \wedge q(y) \implies f(x) = g(y)$.
