\subsection{Encoding composition of predicates and functions}
As the encodings in type encoding are static, that meaning
that the context of every predicate type is fixed
at compile time, QuickSpec will not find equations
that look like $P(f(x))\implies g\;x=h\;x$
unless we have explicitly included the 
predicate $P \circ f$. This property of type encoding
forces us to take in to consideration before running QuickSpec
what functions we are interested in. 
Furthermore, to decrease the size of the resulting expressions we introduce
a variant on type encoding where both the original argument(s) and the
functions before the predicate are kept in the type
\begin{verbatim}
-- Function
f :: A -> B
-- Predicate
p :: B -> Bool

-- Variant on predicate type
data PredicatePF = PF {x :: A, fx :: B}

-- Arbitrary instance
instance Arbitrary PredicatePF where
    arbitrary = do
                  x <- arbitrary `suchThat` (p . f)
                  return (PF x (f x))

\end{verbatim}
This is primarily usefull for expressions like $P (f (g x))\implies h(f(g(x))) = t x$
where we now represent $f(g(x))$ as $fg\;x$, which means QuickSpec is still able to
deal with the function $f\circ g$ without the size of the expression growing.

For predicates of higher arity we get
\begin{verbatim}
-- Functions
f :: A -> B
g :: C -> D

-- Predicate
P :: B -> D -> Bool

data PredicatePFG = PFG {x :: A, fx :: B, y :: X, gy :: D}
\end{verbatim}

As it is tedious work to create these types one idea would be to
simply have the user input the depth or size
of the extended predicates and to generate every possible well typed
extended predicate up to that depth or size. However, as one may expect
this introduces some serious overhead, consider the small QuickSpec signature
\begin{verbatim}
-- Functions
reverse :: [a] -> [a]
length  :: [a] -> Int

-- Constants
0 :: Int
1 :: Int

-- Predicates
(>=)     :: Int -> Int -> Bool 
isSorted :: (Ord a) => [a] -> [a] -> Bool
\end{verbatim}
Using this method and specifying maximum size of the extended predicates as 4
we see that we all ready have a problem. Here are some of the more amusing predicates
\begin{verbatim}
\xs -> (>=) (length xs) 0 -- Any list
\xs -> (>=) 0 (length xs) -- The empty list
\xs -> isSorted (reverse (reverse xs)) -- isSorted
\end{verbatim}
In order to avoid this we can run QuickSpec once with only the standard signature (perhaps even with
normal predicates) and then using the resulting theory we do not include an extended predicate
that QuickSpec consideres equivalent to a smaller extended predicate.
Running QuickSpec twice may seem like it would introduce a significant amount of overhead. However,
as we only need to consider terms of the same size as the size specified by the user for the
extended predicates we do not need to run it for the whole theory, which significantly reduces the overhead
from old QuickSpec's version of conditonal equations.
