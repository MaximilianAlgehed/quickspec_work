\section{Compose I wanted to do this...}
Composition of a function and a predicate
is a very interesting situation. Because the
encodings in type encoding are static, that meaning
that the context of every predicate type is fixed
at compile time, QuickSpec won't find equations
that look like $P(f(x))\implies g\;x=h\;x$
unless we have explicitly stated that we are interested
in the extended predicate $P \circ f$. This property of type encoding
(and as it turns out of function and by extension combined encoding too)
forces us to take in to consideration before running QuickSpec
what functions we are interested in. %Here we need to write something about higher arity predicates and functions
Because this is tedious work one idea would be to simply have the user input the depth or size
of the extended predicates and to generate every possible well formed
extended predicate up to that depth or size. However, as one may expect
this introduces some serious overhead, consider the QuickSpec small signature:
\begin{verbatim}
-- Functions
reverse :: [a] -> [a]
length  :: [a] -> Int

-- Constants
0 :: Int
1 :: Int

-- Predicates
(>=)     :: Int -> Int -> Bool 
isSorted :: (Ord a) => [a] -> [a] -> Bool
\end{verbatim}
Using this method and specifying maximum size of the extended predicates as 4
we see that we all ready have a problem. Here are some of the more amusing predicates:
\begin{verbatim}
(\xs -> (>=) (length xs) 0) -- Any list
(\xs -> (>=) 0 (length xs)) -- The empty list
(\xs -> isSorted (reverse (reverse xs))) -- isSorted
\end{verbatim}
