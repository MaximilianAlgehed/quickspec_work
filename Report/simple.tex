\section{Finding simple conjectures}

This section aims to convey the results which
all three methods have in common and serves as an overview
of the things QuickSpec is capable of doing when augmented with
the ability to generate conditional equations.

All three methods described in section \ref{encodings}
are capable of expressing conditional equations on the form
\\$\forall x_0, x_1, ..., x_n, i, j,.\;(i \neq j \implies x_i \neq x_j)\wedge P_0(x_0) \wedge P_1(x_1) \wedge ... \wedge P_n(x_n) \implies f(x_j, x_k ...) = g(x_a, x_b, ...)$.\\
This means that QuickSpec is now capable of finding several
interesting conjectures.

\subsection{A gentle start} \label{bitvector}
    Bitvectors, expressed neatly as lists of booleans,
    have the following nice property
    \begin{gather*}
        P(n:\mathbb{N},\;xs:List\;\mathbb{B}) = "first\;n\;values\;equal\;False"\\
        \forall n:\mathbb{N},\;xs:List\;\mathbb{B}.\;P(n,\;xs)\implies lsr\;n\;(lsl\;n\;xs)=xs
    \end{gather*}
    Using either of the three encodings presented in section \ref{encodings} QuickSpec
    is able to find this conjecture.

\subsection{Lists, lists, and more lists...}
    There are many interesting conditional equations that
    QuickSpec can discover about lists. Here is a subset of them
    \begin{gather*}
        \forall n:\mathbb{N},\; xs, is:List\; a.\\
        n \leq length\; xs \implies append\; (drop\; n\; xs)\; is\; =\; drop\; n\; (append\; xs\; is)\\\\
        \forall xs, is:List\; a.\\
        xs \neq Nil \implies head\; (append\; xs\: is) = head\; xs\\\\
        \forall xs,\; ys:List\; a.\\
        length\; xs = length\; ys \implies reverse\; (zip\; xs\; (reverse\; ys)) = zip\; (reverse\; xs)\; ys
    \end{gather*}

\subsection{The prettiest printing}
    When exploring Hughes pretty printing DSL,\cite{Hughes1995}
    a task which QuickSpec has done well at before%reference?
    , we now find a law that QuickSpec
    was not previously able to explicitly express
    \begin{gather*}
        P(x:Document) = "x\; is\; not\; indented"\\
        \forall x:Document.\; P(x)\implies text\;Nil\diamond x=x
    \end{gather*}
    The fact that the equation\\
    $\forall x:Document.\; text\;Nil\diamond x = x$\\
    was not in the output from QuickSpec tells us that there is a special
    relationship between unindended documents to the $text$ function. In fact, this relationship
    is that the partially applied function $text\;Nil\;\diamond$ unindents any document it is applied to. 

\subsection{A magic trick}
    In \cite{Claessen2010}Claessen et. al.
    use QuickSpec to find the implementation of the leftist
    heap $insert$ and $deleteMin$ operations. 
    We are going to do a similar thing. When given the predicate
    $>$ and the functions $insert$ and $insertionSort$ QuickSpec gives us
    the rather lovely set of conditional equations
    \begin{gather*}
        \forall x,\;y:a,\; xs:List\; a.\\
        x > y \implies insert\; x\; (Cons\; y\; xs) = Cons\; y\; (insert\; x\; xs)\\\\
        \forall x,\;y:a,\; xs:List\; a.\\
        x > y \implies insert\; y\; (Cons\; x\; xs) = Cons\; y\; (Cons\; x\; xs)\\\\
        insertionSort\; Nil = Nil\\\\
        \forall x:a,\; xs:List\; a.\\
        insertionSort\; (Cons\; x\; xs) = insert\; x\; (insertionSort\; xs)
    \end{gather*}
    which are in fact the complete definition of insertion sort.
    When given the predicate $sorted$ QuickSpec also finds
    $\forall x:a,\; xs:\;List\;a.\; sorted\; xs \implies sorted\; (insert\; x\; xs)$,
    an important lemma needed to inductively prove the correctness of insertion sort.
