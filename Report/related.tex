\section{Related work}
An old version of QuickSpec generated conditional
equations by running QuickSpec once per possible precondition.
This work improves on that approach by integrating the
discovery of conditional equations in to the tool
in the way suggested in the upcoming paper about
QuickSpec.
The old method for condtionals in QuickSpec can 
not take advantage of QuickSpec's schema system.
The method would generate equations like
\begin{equation}\label{eq:old1}
    \forall x, y.\;P(x)\implies f(x) = g(y)\\
\end{equation}
\begin{equation}\label{eq:old2}
    \forall x, y.\;P(x)\wedge\;Q(y)\implies f(x) = g(y)
\end{equation}
As (\ref{eq:old2}) follows from the $\forall y$ in (\ref{eq:old1})
the schema system would not specialize (\ref{eq:old1}) to
(\ref{eq:old2}), which is a clear advantage of
our current method to the previous one.
