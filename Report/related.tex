\section{Related work}
An old version of QuickSpec generated conditional
equations by running QuickSpec once per possible precondition.
This work improves on that approach by integrating the
discovery of conditional equations in to the tool
in the way suggested in the upcoming paper about
QuickSpec.
The old method for condtionals in QuickSpec can 
not take advantage of QuickSpec's schema system.
The method would generate equations like
\begin{equation}\label{eq:old1}
    \forall x, y.\;P(x)\implies f(x) = g(y)\\
\end{equation}
\begin{equation}\label{eq:old2}
    \forall x, y.\;P(x)\wedge\;Q(y)\implies f(x) = g(y)
\end{equation}
As (\ref{eq:old2}) follows from the $\forall y$ in (\ref{eq:old1})
the schema system would not specialize (\ref{eq:old1}) to
(\ref{eq:old2}), which is a clear advantage of
our current method to the previous one.

Other theory exploration and lemma
discovery tools, like the one described in
\cite{Buchberger2001} which uses static rules
to generate new conjectures from existing
lemmas, or \cite{heras2013} which uses
machine learning to do the same thing, 
commonly rely on the existence of old lemmas
to generate new conditions. These tools generally
are not limited to equational reasoning in the way that QuickSpec
is and as such are able to generate conditional lemmas.
However, they are dependent on the existance of
previous lemmas. QuickSpec does not have this dependency.

The schema system in QuickSpec is based on the 
work on IsaCosy described in \cite{Johansson2011}. One advantage
QuickSpec has over IsaCosy is speed, our approach brings
conditional conjectures to QuickSpec without incurring
a very large runtime overhead. QuickSpec is still orders of
magnitude faster than IsaCosy with the most naive (machine generated)
type encodings described in this report.
