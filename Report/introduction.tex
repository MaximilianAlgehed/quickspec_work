\section{Introduction}
Theory exploration tools are programs that
automatically build rich theories about a given mathematical
domain. %something about usefullness  
Existing theory exploration tools have limited or no
support for automatically discovering conditional equations.
%Nothing is mentioned in:
%Scheme-based theorem discovery and concept invention
%Hipster paper obviously
This work builds on the work of Johansson et. al. %Hipster paper reference here
by providing a theoretical framework for introducing conditional
equations in the lightweight theory exploration tool QuickSpec. %Quickspec paper
QuickSpec generates equational conjectures and uses random testing to
find equations that may be valid. QuickSpec uses a schema system to
generate equations and as such is limited to finding equational conjectures.
This work aims to compliment QuickSpec by introducing three different methods
of encoding conditional conjectures as equational conjectures and comparing
the usefullness of these three methods. QuickSpec can be used both as a standalone
system allowing the user to investigate a set of combinators and datatypes in the
purely functional lazy programming language Haskell and as a component of other tools %haskell reference
like tip-spec and HipSpec. %TIP and Hipspec references
Familiarity with QuickSpec is assumed in the rest of this report.
