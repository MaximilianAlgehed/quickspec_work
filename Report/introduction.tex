\section{Introduction}
Types with invariants are the key to
introducing conditional conjectures to
the theory exploration tool QuickSpec.
Theory exploration is the process of
finding plausible conjectures for use in
automated theory provers\cite{Johansson2014}. 
Theory exploration tools support automatically generating
conjectures for the prover to try to prove correct,
these systems also commonly employ some form of lemma discovery process\cite{heras2013}.
This work builds on the work of \cite{Johansson2014} and \cite{Claessen2010}
by providing a theoretical framework for introducing conditional
equations in the light weight theory exploration tool QuickSpec\cite{Claessen2010}.
QuickSpec generates equational conjectures and uses random testing via the tool QuickCheck\cite{Claessen2000}
to find equations that may be valid. It uses testing
to find equivalent haskell terms. Because terms are
tested for equivalence it only finds equational conjectures.

When using QuickSpec to generate lemmas about the functions
\begin{verbatim}
zip     :: [a] -> [b] -> [(a, b)]
reverse :: [a] -> [a]
(++)    :: [a] -> [a] -> [a]
length  :: [a] -> Int
\end{verbatim}
It returns (a superset of) these equations
\begin{verbatim}
zip is (reverse is ++ js)   = zip is (reverse is)
zip (is ++ js) (reverse is) = zip is (reverse is)
zip is (js ++ (is ++ ks))   = zip is (js ++ is)
zip (is ++ (js ++ ks)) js   = zip (is ++ js) js
\end{verbatim}
Here we see that QuickSpec finds equations that are true
because of the way the length of the two lists given to zip relate
to each other. As an example, QuickSpec has generated the equation
\begin{verbatim}zip (is ++ js) (reverse is) = zip is (reverse is)\end{verbatim}
because zip has the property \begin{verbatim}length xs = length ys => zip xs (ys ++ is) = zip xs ys\end{verbatim}
QuickSpec finds the weaker equations but fails to find the general form
of these equations because it has no ability to reason about
conditional equations.
Using our contributions we are able to generate (a superset of) the following conditional
equations in the above example
\begin{verbatim}
length xs = length ys => zip (xs ++ is) ys   = zip xs ys
length xs = length ys => zip ys (xs ++ is)   = zip ys xs
length xs = length ys => reverse (zip xs ys) = zip (reverse xs) (reverse ys)
\end{verbatim}
This work compliments QuickSpec by introducing two different methods
of encoding conditional conjectures as equational conjectures showing that
the method of encoding predicates as types with invariants is more efficient
and applicable than the two alternatives. QuickSpec can be used both as a standalone
system allowing the user to investigate a set of combinators and datatypes in the
purely functional lazy programming language Haskell and as a component of other tools %haskell reference
like TIP\cite{Rosen2015} and HipSpec\cite{Claessen2013}. 
All the work in this report concerns strategies and extensions to the way a user would
use QuickSpec. No changes have been made to the code of QuickSpec in order to
achieve the results described here. This hints that the approach is general
and may be applicable to tools with similar constraints as QuickSpec.

The contributions of this report are
\begin{itemize}
    \item We take the old idea of encoding predicates as types
        in property based testing and show that it is usefull in
        the context of testing based conjecture discovery.

    \item We show that encoding predicates as function in the context
        of QuickSpec compares unfavorably to the laternative method
        of using types.

    \item We show that type encoded predicates can be employed handle
        patrial functions that appear in a shallow embedding of a language
        we want to investigate using QuickSpec.
\end{itemize}
