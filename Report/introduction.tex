\section{Introduction}
Theory exploration tools are programs that
explore a mathematical domain to aid users in
understanding the domain. %something about usefullness  
These tools are primarily theorem provers with a focus
on proving many theorems in a domain of discourse.
Some theory exploration tools support automatically generating
interesting conjectures for the prover to try to prove correct.
However, existing theory exploration tools have limited or no
support for automatically discovering conditional equations. %Sources showing this
This work builds on the work of Johansson et. al. %Hipster paper reference here
by providing a theoretical framework for introducing conditional
equations in the light weight theory exploration tool QuickSpec. %Quickspec paper
QuickSpec generates equational conjectures and uses random testing via the tool QuickCheck %%QuickCheck reference
to find equations that may be valid. QuickSpec currently only generates
equational terms for testing and as such is only capable of generating
theorems that take the form of equations. 
This work aims to compliment QuickSpec by introducing three different methods
of encoding conditional conjectures as equational conjectures and comparing
the usefullness of these three methods. QuickSpec can be used both as a standalone
system allowing the user to investigate a set of combinators and datatypes in the
purely functional lazy programming language Haskell and as a component of other tools %haskell reference
like tip-spec and HipSpec. %TIP and Hipspec references
All the work in this report concerns strategies and extensions to the way a user would
use QuickSpec. No changes have been made to the code of QuickSpec in order to
achieve the results described here. This hints that the approach is general
and may be applicable to tools with similar constraints as QuickSpec.

The contributions of this report can be summarised as
\begin{itemize}
    \item We take the old idea of encoding predicates as types
        in property based testing and show that it is usefull in
        the context of testing based lemma discovery.

    \item We show that encoding predicates as function in the context
        of QuickSpec compares unfavorably to the laternative method
        of using types.

    \item We show that type encoded predicates can be employed to avoid 
        the need for dependently typed shallow embaddings of a language
        we want to investigate using QuickSpec.
\end{itemize}
