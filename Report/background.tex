\section{Background}
In this section we give a brief overview of
QuickSpec.
QuickSpec(2) is a powerfull conjuecture generation
tool that takes as input a set of functions and constants
and produces a set of probably equal terms.
Function symbols in QuickSpec are opaque, the tool
does not know anything about the definition of
the functions other than their type signature.
It employs a schema system and Knuth-Bendix
completion and is based in spirit on the old
tool QuickSpec\cite{Claessen2010}.
The tool generates all terms up to a given
size by enumarating generalized meta-terms called
schemas. These schemas all represent
equations. QuickSpec finds equations by instantiating
schemas with variables and testing them for equality
using the property testing tool QuickCheck\cite{Claessen2000}.

QuickSpec is used as a component of the inductive
theory provers HipSpec\cite{Claessen2013} and Hipster\cite{Johansson2014}.
In these tools QuickSpec is used to generate
a background of conjectures for the theory provers
to consider.
