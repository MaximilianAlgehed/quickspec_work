The theory exploration tool QuickSpec
supports only equational reasoning without
conditional equations. Support for conditional
equations in the related tool HipSpec is limited.
This report discusses the idea of embedding conditional
equations in a setting which only supports normal equations.
We present a comparison of three ways of introducing
conditional equations to QuickSpec.
Our analysis shows that encoding conditions as types with invariants 
yields the best results and we present an overview of
the challenges associated with implementing
this approach as well as examples
of it's power and limitations.
