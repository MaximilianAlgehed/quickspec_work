\section{A programming language}\label{apl}
The APL programming language %apl reference
is a language for processing array and tabular data.
As a language specialized for this task it has a set of very powerfull
operators which are interesting to study and understand.
To work with APL in Haskell we implement a small
algebra for Haskell's vector type \footnote{\texttt{http://hackage.haskell.org/package/yi-0.7.0/docs/Data-Vector.html}}
that emulates the basic semantics of APL.
Using conditionals in QuickSpec we are able to discover
equations about APL in spite of some of the
difficulties of modelling a dynamically typed language
in which some operators can be modelled with dependent types.

Partiality of shallow embeddings representing malformed expressions in the
embedded language can be handled by conditional equations in QuickSpec.
Take for an instance the unary operator $\iota$ in APL. The semantics
of $\iota$ are that for any dimension matrix which contains only one element, which is also
an integer, the $\iota$ operator will produce the vector of all integers from 1 to that element.
The implementation of the $\iota$ operator in a shallow embedding will be inherently partial
but since we can deal with partial functions using our techniques we are still able to reason
about equations involving $\iota$.

The unary $\rho$ operator in APL takes the shape of it's operand,
so $\rho$ of a scalar is \texttt{[]} 
and $\rho$\texttt{[0, 1, 3] = [3]} and
$\rho\circ\rho$ is a function that gives the dimension of the operand (scalars having dimension 0).
A property of this operator is that it requires the operand to be homogenious.
This means that in APL the operation $\rho$\texttt{[[1, 2], [1, 3, 2]]}
would result in an error because the first and second sub-vector do not have the same shape.
This property of $\rho$ becomes very relevant when we consider operators like $\times$ - element wise
multiplication. In the special (but most common and obvious) case
when the $\rho\circ\rho$ of the two operands agree the $\times$ operation
is only defined if the two operands also have the same element-wise $\rho$. So, for two "well bahaved" matrices
$x$ and $y$ we would expect the property $\rho(x \times y) = \rho x = \rho y$ to hold. Furthermore
we would expect this property to hold for many operations like element wise subtraction, addition, equality etc..
The good news is that QuickSpec is able to find these (conditional!) equations using type encoding. However,
as discussed in section \ref{conjunction} we need to define a custom generator for these predicate types
and as such they are not very general.
