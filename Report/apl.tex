\section{A programming language}
The programming language APL %apl reference
is a language for processing arrays and tabular data.
APL has very concise notation and semantics.
To work with APL in haskell we implement a small
algebra for haskell's vector type %reference to vector (?)
that emulates the basic semantics of APL.
Using conditionals in QuickSpec we are able to discover
equations about APL in spite of some of the
difficulties of modelling a dynamically typed language that 
behaves like a dependently typed duck.

An example of this is matrices in APL. Matrices in APL
have the same properties of rank that those of normal
matrices in standard linear algebra. However, in our implementation
of matrices in haskell one may choose to represent a two dimensional
APL matrix with the type:
\begin{verbatim}type APLMatrix a = Vector (Vector a)\end{verbatim}
Using conditional equations in QuickSpec we are still able to deal with
APL matrices even though our type has none of the properties of APL matrices.
One such property, our type involves the value \begin{verbatim} [[], [1, 2, 3]] \end{verbatim}
which is not a valid value in APL.

Partiality of shallow embeddings representing malformed expressions in the
embedded language can be represented by conditional equations in QuickSpec.
Take for an instance the unary operator $\iota$ in APL. The semantics
of $\iota$ are that any dimension matrix which contains only one element which is also
an integer the $\iota$ operator will produce the vector of all integers from 1 to that element.
The implementation of the $\iota$ operator in a shallow embedding will be inherently partial,
but since we can deal with partial functions using our techniques we are still able to reason
about equations involving $\iota$.
