\section{A programming language}\label{apl}
The APL programming language\cite{Iverson1962}\cite{APLdic}
is a language for processing array and tabular data.
To work with APL in Haskell we implement a small
set of combinators for Haskell's vector
type\footnote{\texttt{http://hackage.haskell.org/package/yi-0.7.0/docs/Data-Vector.html}}
that emulates the basic semantics of APL.
Using conditional equations in QuickSpec we are able to discover
equations about APL which QuickSpec would not
normally find as a lot of functions in our
shallow embedding of APL are partial\cite{Claessen2010}.

The unary $\rho$ operator in APL takes the shape of it's operand.
For example \\$\rho$\texttt{[1, 2]} is \texttt{[2]} and
$\rho$\texttt{[[0], [1], [3]] = [3, 1]}.
$\rho\circ\rho$ gives the dimension of the operand (scalars have dimension 0).
A property of this operator is that it requires the operand to be homogenious.
This means that in APL the operation $\rho$\texttt{[[1, 2], [1, 2, 3]]}
would result in an error because the first and second sub-vector do not have the same length.
This property of $\rho$ is relevant when considering operators like $\times$ - element wise
multiplication, $\times$ requires the two operands to have the same shape
However $\rho$ does not need to be defined for the operands. For two "well bahaved" matrices
$x$ and $y$ we would expect the property $\rho(x \times y) = \rho x = \rho y$ to hold. Furthermore
we would expect this property to hold for many operations like element wise subtraction, addition, equality etc..
QuickSpec is able to find these equations using type encoding. However, as discussed
in section \ref{conjunction} we need to define a custom generator for these predicate types
and as such they are not very general.
