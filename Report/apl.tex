\section{A programming language}
The programming language APL %apl reference
is a language for processing arrays and tabular data.
APL has very concise notation and semantics.
To work with APL in haskell we implement a small
algebra for haskell's vector type %reference to vector (?)
that emulates the basic semantics of APL.
Using conditionals in QuickSpec we are able to check some of
the semantics of APL at runtime.

An example of this is matrices in APL. Matrices in APL
have the same properties of rank that those of normal
matrices in standard linear algebra. However, in our implementation
of matrices in haskell one may choose to represent an APL matrix with
the type:
\begin{verbatim}type APLMatrix a = Vector (Vector a)\end{verbatim}
Using conditional equations in QuickSpec we are still able to deal with
APL matrices even though our type has none of the properties of APL matrices.
One such property, our type involves the value \begin{verbatim} [[], [1, 2, 3]] \end{verbatim}
which is not a valid value in APL.
