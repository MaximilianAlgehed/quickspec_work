\section{Injective functions in Type Encoding}
When trying to prove injectivity of a pretty printer for arithemtic expressions one
finds oneself needing the following predicate (1) and associated lemma (2):
\begin{verbatim}

P(e, f, s, t) := showTerm e ++ s = showTerm f ++ t (1)
P(e, f, s, t) => e = f && s = t (2)

\end{verbatim}
When adding P to the signature of QuickSpec however, the tool grinds to a complete halt. This is
due to, as we can see in (2), generating data where P holds implies generating two
random expression which are equal, and generating two random strings which are also equal.

The problem here is that \texttt{showTerm} and the partially applied append function are both injective.
Our experiments show that predicates involving equality and injective functions generaly do not
fare well in a normal type encoding setting.

One idea for how to deal with injective functions is to let the user specify that they suspect that 
a function may be injective. This would then let us write a generator that is more likely to generate
equal data. One could even imagine having a generator which chooses between arbitrary data
and data which is very likely to be equal. This way corner cases where the function is not injective may
still be found.
