\subsection{Injective functions in Type Encoding}
When trying to prove injectivity of a pretty printer for arithemtic expressions one
finds oneself needing the following predicate (1) and associated lemma (2):
\begin{verbatim}
P(e, f, s, t) := show e ++ s = show f ++ t (1)
P(e, f, s, t) => e = f && s = t (2)
\end{verbatim}
When adding P to the signature of QuickSpec however, the tool grinds to a complete halt. This is
due to, as we can see in (2), generating data where P holds implies generating two
random expression which are equal, and generating two random strings which are also equal.

For the sake of simplicity we consider only
\begin{verbatim}
P(e, f) := show e = show f
P(e, f) => e = f
\end{verbatim}

The problem here is that \texttt{showTerm} and the partially applied append function are both injective.
Our experiments show that predicates involving equality and injective functions generaly do not
fare well in a normal type encoding setting. The reason for this is that naive type encoding uses
QuickCheck's \texttt{suchThat} combinator, which works by incrementing the size parameter of the QuickCheck
generator and testing generated values untill the predicate holds.

One idea for how to deal with injective functions is to let the user specify that they suspect that 
a function may be injective. This lets us create a generator that is more likely to generate
equal data. One naive generator for the predicate in the above example is the following
\begin{verbatim}
predicate :: Expression -> Expression -> Bool
predicate v w = show v == show w

data P = P {a21 :: Expression,
            a22 :: Expression
           }

instance Arbitrary P where
    arbitrary = arb `suchThat` (\p -> predicate (a21 p) (a22 p))
                where
                    arb = do
                            e <- arbitrary
                            e' <- oneof $ [return e, arbitrary]
                            return (Ps e e')
\end{verbatim}
With this generator QuickSpec finds the equation \texttt{P(e, f) => e = f}.

However, this is only part of the truth. If we load the expression module in to ghci and
play around with it we find
\begin{verbatim}
ghci> show (X :+: (X :+: X))
    "x + x + x"
ghci> show ((X :+: X) :+: X)
    "x + x + x"
\end{verbatim}
So \texttt{show} is not at all injective, QuickSpec was wrong! Why did this happen?
The problem is that the type for expressions is very big, and generating the exact pathological
example that breaks \texttt{show} randomly is highly unlikely. Since in this example we are testing
for bugs in the \texttt{show} function one might expect that if there are bugs in the code then
they will manifest when we do small mutations to the abstract syntax tree of expressions, such as
reassociating \texttt{X :+: (X :+: X)} to \texttt{(X :+: X) :+: X}.
