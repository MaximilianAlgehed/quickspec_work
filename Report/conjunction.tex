\section{Conjunction}\label{conjunction}
Consider the bit-vector example from section
\ref{bitvector}, the conditional equation
\\$\forall n:\mathbb{N},\;xs:List\;\mathbb{B}.\;P(n,\;xs)\implies lsr\;n\;(lsl\;n\;xs)=xs$\\
can be generalized to
\\$\forall n,m:\mathbb{N},\;xs:List\;\mathbb{B}.\;(m\leq n)\wedge P(n,\;xs)\implies lsr\;n\;(lsl\;n\;xs)=xs$.

Conjunction is necessary for expressing conditional equations after invariants.
As evidenced by section \ref{apl}, where we find that one predicate, "is well shaped",
is used to express invariants of the language and another predicate, "shape of these two matrices
are equal", is used to express a property of interest in the domain of APL.
This property is not unique to shallow embeddings of APL, rather it shows up in several situations.
Consider sets represented as red-black trees, to find proper conjectures about
disjoint sets we would have to have two predicates to describe the well-formedness
and disjointness of two sets. One approach to solving the problem of conjunction of predicates over this
type of data would be to do what QuickCheck does\cite{Claessen2000}; define instances
of arbitrary that mean the set datastructure always has the property of a set (if the set is represented
as a red-black tree for an instance these properties would be precisely the invariants of red-black trees). 
However, this method requires the user to define the generator, not just the predicate and the datatype.

Using the method of automatically generated predicate types in type encoding, described in section
\ref{TE}, we find that conjunction becomes impossibly slow. To see why this is we need to look at the
details of how we go about representing conjunction of predicates in type encoding. Recall that
binary predicates can be naively represented as datatypes that look something like this:
\begin{verbatim}data Predicate = Predicate {data0 :: A, data1 :: B}

instance Arbitrary Predicate where
    
    arbitrary = fmap (\(a, b) -> Predicate a b) $
                arbitrary `suchThat` (\(a, b) -> p a b)

\end{verbatim}
Here the type \texttt{Predicate} represents the binary predicate \texttt{p}.
To work with conditional equations like the ones above that look like
\\$P(x) \wedge P(y) \wedge Q(x, y)\implies f\;x\;y=g\;x\;y$\\
we can do lambda abstraction over the predicates and replace them with
one predicate $H$. At this point we stop to consider the complexity of $H$. Not only
does it contain two conjunctions, which are likely to be represented as normal (non-parallel) $\wedge$,
it is also highly unlikely that our automatic "general" method for encoding conditionals
will work. This is due to the complexity of the predicate, \texttt{suchThat}'s guessing strategy
is very unlikely to find two elements for which $H$ will hold.

%Somewhere around here is where we introduce the ideas from lazy feat or whatever
