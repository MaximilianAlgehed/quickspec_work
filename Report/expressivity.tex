\section{Expressivity}

A question which naturally arises when studying the
definitions in section \ref{encodings} is;
Can every equation that is expressed in a specific encoding
scheme really be translated into a conditional equation?
The answer to this question, sadly, is no. The semantics
for how one would translate a function encoding or a combined
encoding to a conditional are not as obvious as one would think.
Consider the following example:
\begin{verbatim}
%Example involving whenBitZeroZero (whenBitSevenZero b) = whenBitSevenZero (whenBitZeroZero b)
%which shows that if we choose repeated application of different predicates
%to mean conjunction of predicates we get problems with
%whenSorted (whenNotNull xs) = f xs
%which can't be translated to
%(whenSorted xs) and (whenNotNull xs) => xs = f xs
\end{verbatim}

The coulprit here is surely the choice of default value for our function encodings. 
In section \ref{combined_encoding} we introduced the idea of combined encoding which was meant to do away
with issues like this. However, as we see in the following example
the semanitcs for combined encoding are not clear either
\begin{verbatim}
%Some example, not sure which it will be yet
\end{verbatim}
