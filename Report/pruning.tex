\subsection{Pruning}
QuickSpec uses a pruner to determine if
it is going to test an equation or if the equations
can be proved from the equations it allready knows.
The normal QuickSpec pruner can not handle type encoding.
To see why consider the following example
\begin{verbatim}
-- Type encoding of sorted list
data Psorted a = P {xs :: [a]}

-- Signature
sorted :: (Ord a) => [a] -> Bool
insert :: (Ord a) => a -> [a] -> [a]
isort  :: (Ord a) => [a] -> [a]

-- Tested, true, equations
sorted (xs p) = True
sorted (isort as) = True
sorted (insert a (xs p)) = True
sorted (insert a (isort as)) = True
\end{verbatim}
Here we see that QuickSpec has tested the term
\\\texttt{sorted (insert a (isort as))} (\texttt{= True} omitted)
even though we can deduce it from
\\\texttt{sorted (isort as)},
\texttt{sorted (insert a (xs p))} and
\texttt{sorted (xs p)}. As it does not follow
equationally. However, if we introduce
a rule to the QuickSpec pruner that says;
whenever there is a term like \\\texttt{sorted as},
add the following equation to the theory
\texttt{xs (toP as) = as}. This now gives us
the equation \texttt{xs (toP (isort as)) = isort as},
where \texttt{toP} is an undefined function with type
\texttt{toP :: [a] -> PSorted a} that is only used to aid the proof,
which gives us, through \texttt{sorted (insert a (xs p))},
the term \texttt{sorted (insert a (isort as))} without
having to test it.

The method can be generalized to higher arity predicates.
One such predicate is \texttt{p xs ys = length xs == length ys}.
With normal type encoding without any support for pruning
conditional equations QuickSpec would test both
\\\texttt{zip (xs p) (ys p ++ zs) = zip (xs p) (ys p)}\\
and \texttt{zip (ys p) (xs p ++ zs) = zip (ys p) (xs p)}, and find
both to be valid.
QuickSpec will also find \texttt{length (xs p) = length (ys p)},
Enhancing QuickSpec's pruner in this case would be introducing
the rule that\\\texttt{length as = length bs} means we can add the equations
\\\texttt{xs (toP as bs) = as} and \texttt{ys (toP as bs) = bs},\\
which are enough to conclude\\
\texttt{zip (xs p) (ys p ++ zs) = zip (xs p) (ys p)}$\implies$
\\\texttt{zip (ys p) (xs p ++ zs) = zip (ys p) (xs p)}.\\As can be
seen below

\texttt{forall p, zs.}\\
\texttt{length (ys p) = length (xs p)}$\implies$\\
\texttt{xs (toP (ys p) (xs p)) = ys p}\\
\texttt{ys (toP (ys p) (xs p)) = xs p}\\\\
\texttt{let q = toP (ys p) (xs p)}\\\\
\texttt{zip (xs p) (ys p ++ zs) = zip (xs p) (ys p)}$\implies$\\
\texttt{zip (xs q) (ys q ++ zs) = zip (xs q) (ys q)}$\implies$\\
\texttt{zip (ys p) (xs p ++ zs) = zip (ys p) (xs p)}

The method can be generalized to any arity predicate.
Take any predicate,\\\texttt{p :: T1 -> ... -> TN -> Bool},
it has the corresponding predicate type
\\\texttt{data P = P \{x1 :: T1, ..., xn :: TN\}},
thus the rules we can introduce to the pruner are
\\\texttt{
p a1 ... an =>\\
x1 (toP a1 ... an) = a1\\
...\\
xn (toP a1 ... an) = an\\
}
